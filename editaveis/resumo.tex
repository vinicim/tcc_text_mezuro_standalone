\begin{resumo}

[Inserir resumo... Exemplo Daniel]

O objetivo desse trabalho de conclusão de curso é apresentar uma justificativa
teórica e levantar os requisitos necessários para viabilizar a implantação
de uma rede de colaboração para a Universidade de Brasília que atue como um
ambiente virtual para a criação e o compartilhamento de conhecimento de forma
colaborativa e horizontal a ser utilizada como uma ferramenta complementar de
apoio à educação.  
%
Esta abordagem permite quebrar a verticalização da relação professor-aluno e
faz com que o aluno passe a atuar mais como autor ou co-autor de conhecimento. 
%
Neste contexto, escolhemos utilizar a plataforma
brasileira para redes sociais livres Noosfero por entender que esta satisfaz as
necessidades imediatas do projeto. Além disso, queremos aproveitar nossa
posição geográfica favorável para contribuir com a comunidade da mesna, uma vez
que a maior parte de seus desenvolvedores estão localizados no Brasil.
%
Pretendemos também colaborar para a implementação de um protocolo de federação
para o Noosfero de forma a permitir a comunicação e a troca de conteúdo entre
outras instâncias deste ou de outras redes que implementem o mesmo protocolo,
tornando-se assim um nó dentro de uma rede de colaboração em potencial 
formada por diversas instituições de ensino, cada qual com sua própria rede.

 \vspace{\onelineskip}
    
 \noindent
 \textbf{Palavras-chaves}: redes sociais. software livre. federação.
\end{resumo}
