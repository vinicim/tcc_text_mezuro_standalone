\begin{resumo}

No contexto deste trabalho está uma plataforma para monitoramento de códigos-fonte chamada Mezuro. Essa plataforma é desenvolvida através de um projeto de software livre e desde sua concepção era um plugin de uma ferramenta para desenvolvimento de redes sociais, o Noosfero.
%
Porém com sucessivas alterações e com o aumento em tamanho e funcionalidades do Mezuro, aumentou-se também sua complexidade, assim como a dificuldade em mantê-la. A equipe de desenvolvimento decidiu então por evoluir essa ferramenta para um aplicação independente.
%
Neste trabalho de conclusão de curso são aprensentados as principais razões e motivações para a evolução dessa plataforma, assim como os impactos em sua arquitetura e nos seus requisitos de qualidade.
%
Apresenta-se também um relato de colaboração com um projeto de software livre, assim como os principais padrões utilizados para tal.

 \vspace{\onelineskip}
    
 \noindent
 \textbf{Palavras-chaves}: software livre. evolução. requisitos de qualidade.
\end{resumo}
