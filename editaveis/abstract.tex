\begin{resumo}[Abstract]
  \begin{otherlanguage*}{english} 
  
[insert abstract... Daniel example]  
    
  This undergraduate work aims to present a theoric justification and to raise
  the requirements needed to enable the development of a collaborative network for
  the University of Brasília.
  %
  This network acts as a virtual environment that allows users to create and share
  knowledge in a collaborative and horizontal way.
  %
  This approach allows to change the "traditional" professor-student relationship
  and promotes students to act more as either an author or co-author of the
  knowledge produced.
  %
  In this context, we selected a brazilian free social networking tool Noosfero.
  %
  It meets the UnB Community project immediate needs. Furthermore, we want to use
  our geographical position to contribute to the community maintaining this plaform
  since it has most of its developers located in Brazil.
  %
  We also intend to collaborate with the development of a federation protocol for
  Noosfero. This feature enables communication and exchange of content between
  other instances of this or other networks that implement the same protocol.
  %
  Thus, it may become a node within a potential collaborative network formed by
  several educational institutions, each with it’s own network.
   

  \vspace{\onelineskip}
 
  \noindent 
  \textbf{Key-words}: social networking. open-source. federation.
  \end{otherlanguage*}
\end{resumo}


