\chapter{Considereções Finais}

A evolução do Mezuro não foi motivada por um motivo isolado. Um conjunto de fatores influenciaram e convenceram a equipe que o melhor para o futuro do projeto Mezuro seria um conjunto de modificações em sua estrutura. Os desenvolvedores estavam restritos aos ultrapassados recursos do Rails 2 e do Ruby 1.8, a qual não recebia mais suporte de seus desenvolvedores.

A equipe almejava por liberdade na tomada de decisões, como por exemplo atualizar o Mezuro para as novas versões do Rails e do Ruby, aproveitando suas melhorias e novos recursos. Porém, estavam subordinados as decisões e andamento do projeto Noosfero. A equipe se deu conta que os recursos relacionados a redes sociais fornecidos não eram necessários para uma ferramenta de monitoramento de código-fonte. Além disso, a manutenbilidade e inserção de novos desenvolvedores ao projeto eram tarefas difíceis, já que o Mezuro crescia bastante e se tornava cada vez mais uma aplicação dentro de outra aplicação, ao invés de um plugin com funcionalidades bem específicas.

Levando em conta todos esses fatores, a equipe do Mezuro decidiu retira-lo do Noosfero, transformando-a em uma aplicação independente, além de formatar uma comunidade de software livre para atrair novos desenvolvedores. Para consolidar esses fatores, que levaram a evolução do Mezuro, foi elaborado um questionário (encontrado no Apêndice \ref{form-pesquisa}) direcionado e equipe que o mantem. Dos fatores que motivaram a evolução, os que mais foram citados pela equipe foram aqueles que limitam sua liberdade ou poder de decisão, que é o fato do Mezuro estar contido dentro do Noosfero, tendo seu desenvolvimento limitado.

A contribuição do principal autor deste trabalho, com a funcionalidade de Manter Repositórios, é um resultado dessa nova comunidade de software livre formada a partir da decisão de tornar a plataforma Mezuro uma aplicação independente. É importante destacar que, embora o Mezuro esteja evoluindo para uma aplicação distinta do Noosfero, haverá uma integração entre essas duas ferramentas futuramente.

\section{Cronograma}

Além da integração do Mezuro com o Noosfero, há outras atividades planejadas para o futuro do Mezuro. Até agora, o Mezuro evoluiu em sua arquitetura, migrando de plugin para aplicação independente com o Rails 4. Nessa segunda fase vamos colaborar com novas funcionalidades. As atividades planejadas são:

\begin{enumerate}
\item Pesquisa sobre visualização de software, formas visuais das métricas de código-fonte
\item Incorporar formas visuais das métricas ao Mezuro
\item Implementar suporte a novas linguagens (e.g Ruby)
\item Integrar Mezuro e Noosfero
\item Escrita do TCC
\end{enumerate}

\begin{table}[H]
\begin{center}
    \begin{tabular}{ | l | l | l | l | l | l | l |}
    \hline
    Atividade & Jan 2014 & Fev 2014 & Mar 2014 & Abr 2014 & Mai 2014 & Jun 2014 \\ \hline
    1 & • & • & & & & \\ \hline
    2 & & • & & & & \\ \hline
    3 & & & • & • & & \\ \hline
    4 & & & & & • & \\ \hline
    5 & & & • & • & • & • \\ \hline
    \end{tabular}
    \caption{Cronograma para atividades do TCC2}
    \label{tab-cronograma}
\end{center}
\end{table}