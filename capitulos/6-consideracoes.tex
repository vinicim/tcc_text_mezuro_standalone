\chapter{Considereções Finais}

A evolução do Mezuro não foi motivada por um motivo isolado. Um conjunto de fatores influenciaram e convenceram a equipe que o melhor para o futuro do projeto Mezuro seria um conjunto de modificações em sua estrutura. Os desenvolvedores estavam restritos aos ultrapassados recursos do Rails 2 e do Ruby 1.8, a qual não recebia mais suporte de seus desenvolvedores.

A equipe almejava por liberdade na tomada de decisões, como por exemplo atualizar o Mezuro para as novas versões do Rails e do Ruby, aproveitando suas melhorias e novos recursos. Porém, estavam subordinados as decisões e andamento do projeto Noosfero. A equipe se deu conta que os recursos relacionados a redes sociais fornecidos não eram necessários para uma ferramenta de monitoramento de código-fonte. Além disso, a manutenibilidade e inserção de novos desenvolvedores ao projeto eram tarefas difíceis, já que o Mezuro crescia bastante e se tornava cada vez mais uma aplicação dentro de outra aplicação, ao invés de um plugin com funcionalidades bem específicas.

Levando em conta todos esses fatores, a equipe do Mezuro decidiu retirá-lo do Noosfero, transformando-a em uma aplicação independente, além de formatar uma comunidade de software livre para atrair novos desenvolvedores. Para consolidar esses fatores, que levaram a evolução do Mezuro, foi elaborado um questionário (encontrado no Apêndice \ref{form-pesquisa}) direcionado e equipe que o mantém. Dos fatores que motivaram a evolução, os que mais foram citados pela equipe foram aqueles que limitam sua liberdade ou poder de decisão, que é o fato do Mezuro estar contido dentro do Noosfero, tendo seu desenvolvimento limitado.

A contribuição do principal autor deste trabalho, com a funcionalidade de   ``Manter Repositórios'', é um resultado dessa nova comunidade de software livre formada a partir da decisão de tornar a plataforma Mezuro uma aplicação independente. É importante destacar que, embora o Mezuro esteja evoluindo para uma aplicação distinta do Noosfero, haverá uma integração, ainda a ser discutida pela comunidade de desenvolvedores, entre essas duas ferramentas futuramente.

\section{Cronograma}

Para o curto e médio prazo, no escopo deste trabalho, há atividades planejadas para o futuro do Mezuro. Até agora, o Mezuro evoluiu em sua arquitetura, migrando de plugin para aplicação independente com o Rails 4. Na segunda fase deste trabalho, vamos colaborar com novas funcionalidades, que também demandará uma etapa de pesquisa complementar ao que fizemos até agora. As atividades planejadas são:

\begin{enumerate}
\item Complementação das pesquisas sobre software livre e evolução de software;
\item Pesquisa sobre visualização de software, formas visuais das métricas de código-fonte;
\item Colaboração com finalização da migração das funcionalidades do Mezuro Plugin para o novo Mezuro;
\item Incorporar formas visuais das métricas ao Mezuro (visualizações gráficas);
\item Implementar suporte a novas linguagens (e.g Ruby);
\item Escrita do TCC.
\end{enumerate}

\begin{table}[H]
\begin{center}
    \begin{tabular}{ | l | l | l | l | l | l | l | l |}
    \hline
    Atividade & Dez 2013 & Jan 2014 & Fev 2014 & Mar 2014 & Abr 2014 & Mai 2014 & Jun 2014 \\ \hline
    1 & • & • &   &   &   &   &   \\ \hline
    2 &   & • & • & • &   &   &   \\ \hline
    3 & • & • & • &   &   &   &   \\ \hline
    4 &   &   &   & • & • & • &   \\ \hline
    5 &   &   &   &   &   & • & • \\ \hline
    6 & • & • & • & • & • & • & • \\ \hline
    \end{tabular}
    \caption{Cronograma para atividades do TCC2}
    \label{tab-cronograma}
\end{center}
\end{table}

É importante enfatizar que, após as leituras preliminares e uma avaliação inicial do Mezuro Plugin para a definição do tema e passos para este trabalho, ao final de agosto de 2013, este trabalho começou a ser desenvolvido em setembro de 2013, desde o aprendizado das tecnologias, além dos estudos teóricos para a elaboração deste texto.
%
Em 2 meses, conseguimos colaborar efetivamente com o Mezuro ao desenvolver toda a parte de ``gestão de repositório'' de um projeto cadastrado no Mezuro, ou seja, um item central entre as funcionalidades do mesmo.
%
Avaliamos que, durante o TCC 1, já percorremos a curva de aprendizado necessária para contribuir com o projeto de Mezuro, para, num primeiro momento, finalizarmos a migração para uma aplicação independente e, posteriormente, inserirmos novas funcionalidades, em especial no contexto de visualização de software, como uma das partes da área de evolução de software que será pesquisada durante o TCC 2 (7 meses, conforme o cronograma apresentado na Tabela \ref{tab-cronograma}).
 

