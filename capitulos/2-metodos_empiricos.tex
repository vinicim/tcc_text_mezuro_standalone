\chapter{Métodos Empíricos}
\label{cap-met-emp}

%------------------------------------------------------------------------------
%inserir introdução do capítulo
%-----------------------------------------------------------------------------

\section{Software Livre}
\label{sec-swl}
%O software, em um sistema computacional, é o componente que contém o conhecimento relacionado aos problemas a que a computação se aplica, contendo diversos aspectos que ultrapassam questões técnicas~\cite{meirelles2013}

O software, em um sistema computacional, é o componente que contém o conhecimento relacionado aos problemas a que a computação se aplica, contendo diversos aspectos que ultrapassam questões técnicas \cite{meirelles2013metrics}, como por exemplo: 

\begin{itemize}
\item O processo de desenvolvimento de software;
\item Os mecanismos econômicos (gerenciais, competitivos, sociais, cognitivos, etc.) que regem esse desenvolvimento e seu uso;
\item O relacionamento entre desenvolvedores, fornecedores e usuários de software;
\item Os aspectos éticos e legais relacionados ao software;
\end{itemize}

O que define e diferencia o software livre de software proprietário vai do entendimento desses quatro pontos dentro do que é conhecido como \textit{ecossistema do software livre}~\cite{meirelles2013}.
O princípio básico desse ecossistema refere à liberdade dos usuários de executar, copiar, distribuir, estudar, mudar e melhorar o software. Estas estão definidas nas quatro liberdades para os usuários do software:

\begin{itemize}
\item A liberdade de executar o programa, para qualquer propósito (liberdade no. 0);
\item A liberdade de estudar como o programa funciona, e adaptá-lo para as suas necessidades (liberdade no. 1). Aceso ao código-fonte é um pré-requisito para esta liberdade;
\item A liberdade de redistribuir cópias de modo que você possa ajudar ao seu próximo (liberdade no. 2);
\item A liberdade de aperfeiçoar o programa, e liberar os seus aperfeiçoamentos, de modo que toda a comunidade se beneficie (liberdade no. 3). Acesso ao código-fonte é um pré-requisito para esta liberdade.
\end{itemize}

Um programa é software livre se os usuários têm todas essas liberdades. Assim, você deve ser livre para redistribuir cópias, seja com ou sem modificações, seja de graça ou cobrando uma taxa pela distribuição, para qualquer um em qualquer lugar.

Você deve também ter a liberdade de fazer modificações e usá-las privativamente no seu trabalho ou lazer, sem nem mesmo mencionar que elas existem. Se você publicar as modificações, você não deve ser obrigado a avisar a ninguém em particular, ou de qualquer forma particular.

A liberdade de executar o programa significa a liberdade para qualquer tipo de pessoa física ou jurídica utilizar o software em qualquer tipo de sistema computacional, para qualquer tipo de trabalho em geral e finalidade, sem que seja necessário comunicar sobre isso com o desenvolvedor ou a qualquer outra entidade em especial.

A liberdade de redistribuir cópias deve incluir formas binárias ou executáveis do programa, assim como o código-fonte, tanto para as versões modificadas.

A liberdade de acesso ao código-fonte do programa. Portanto, acesso ao código-fonte é uma condição necessária ao software livre ~\cite{gnu2013}.

%
Essa liberdade dentro do software livre é dita por Richard Stallman como a principal vantagem. Software não-livre é ruim porque nega sua liberdade. Então, perguntar sobre as vantagens práticas do software livre é como perguntar sobre as vantagens práticas de não estar algemado ~\cite{stallman2009}.

%
O GNU quis dar aos utilizadores a liberdade, não só para ser popular. Contudo precisava usar termos de distribuição que impediriam software livre de ser transformado em software proprietário. O método usado foi chamado de copyleft. Copyleft é um método geral para fazer um software livre e exige que todas as versões modificadas e estendidas do programa sejam também.Ele utiliza lei de direitos autorais, mas veio para servir como oposto de sua finalidade usual: ao invés de um meio de privatizar o software, torna-se um meio de manter o software livre~\cite{stallman2009}.

%---------------------------------------------------------------------------------
%trecho retirado após o merge
%----------------------------------------------------------------------------------

%%Diferenciar software livre de software proprietario
%O entendimento desses quatro pontos é o que diferencia softwares ditos restritos dos livres, bem como, é o que define o que é conhecido como ``ecossitema do software livre''. O princípio básico desse ecossistema é promover a liberdade do usuário, sem discriminar quem tem permissão para usar um software e seus limites de uso, baseado na colaboração e num processo de desenvolvimento aberto \cite{meirelles2013metrics}.
%
%Ao contrário do software restrito, o software livre tem a característica do compartilhamento do seu código-fonte. Essa característica oferece vantagens em relação ao software proprietário. O compartilhamento permite a simplificação de aplicações personalizadas, já que não necessitam serem codificadas do zero, podendo se basear em soluções existentes. Outra vantagem é a melhoria da qualidade \cite{raymond1999cathedral}, por conta da grande quantidade de colaboradores, que com diferentes perspectivas e necessidades, propõem melhorias para o sistema, além de identificar e corrgir bugs com mais rapidez.
%
%Em resumo, software livre representa uma classe de sistemas de software, os quais são distribuídos sob licenças cujos termos permitem aos seus usuários utilizar, estudar, modificar e redistribuir o software \cite{terceiro2012freesoftware}. 

%----------------------------------------------------------------------------------

\section{Métodos Ágeis}
\label{sec-metedos-ageis}

Métodos ágeis (AM) é uma coleção de metodologias baseada na prática para 
modelagem efetiva de sistemas baseados em software. É uma filosofia onde muitas metodologias se encaixam.

%
As metodologias ágeis aplicam uma coleção de práticas, guiadas por princípios e valores que podem ser aplicados por profissionais de software no dia a dia~\cite{}%Procurar referencia.

\subsection{Metodologia Ágil}
\label{metodologia-agil}

A expressão "Metodologias Ágeis" tornou-se popular em 2001 quando dezessete especialistas em processos de desenvolvimento de software representando diversos métodos como Scrum ~\cite{schwaber2002},
Extreme Programming (XP) e outros, estabeleceram princípios comuns compartilhados por todos esses métodos. Foi então criada a Aliança Ágil e o estabelecimento do Manifesto Ágil ~\cite{manifesto2013}
Os conceitos chave do Manifesto Ágil são:
\textbf{Indivíduos e interações} ao invés de processos e ferra-
mentas.
\textbf{Software executável} ao invés de documentação.
\textbf{Colaboração do cliente} ao invés de negociação de contratos.
\textbf{Respostas rápidas a mudanças} ao invés de seguir planos.
O Manifesto Ágil não discrimina processos, ferramentas, documentação, negociação de contratos ou o planejamento, mas simplesmente mostra que eles têm importância secundária quando comparado com os indivíduos, interações, software executável, participação do cliente e feedback rápido a mudanças e alterações.

%
Tory Dyba (\citeyear{dyba2008}) em uma revisão sistemática elenca seis dos principais métodos ágeis presente em 2008 e trás uma breve descrição de cada um, o que ajuda a observar características das comunidades ágeis já que os principais especialistas desses métodos ajudaram a escrever o manifesto ágil. Eles são:

\begin{description}
%\begin{enumerate}

\item [1. Crystal Clear]
%
centra-se na comunicação de pequenas equipes de
desenvolvimento de software que não possui ciclo de vida crítico.
%
Seu desenvolvimento tem sete características: entrega frequente,
melhoria reflexivo, comunicação osmótica, segurança pessoal, foco,
fácil acesso a usuários experientes e os requisitos para o ambiente técnico.

\item [2. Dynamic software development method (DSDM)]
%
divide projetos em três fases: pré-projeto, o ciclo de vida do projeto  e pós projeto. Nove princípios subjacentes ao DSDM: o envolvimento do usuário, capacitando da equipe do projeto,  a entrega frequente, atender às necessidades de negócios atuais,  desenvolvimento iterativo e incremental, permitir mudanças,  o escopo de alto nível a ser fixados antes do início do projeto,  testar todo o ciclo de vida e eficiente e eficaz comunicação.
            
\item [3. Feature-driven development]
%
combina desenvolvimento model-driven e ágil, com ênfase  em modelo inicial de objeto, a divisão do trabalho em funções  e design iterativo para cada recurso. Afirma ser adequado para  o desenvolvimento de sistemas críticos. Uma iteração de um  recurso é composto de duas fases: concepção e desenvolvimento
       
\item [4. Lean software development]
%
uma adaptação de princípios de produção de carne magra e,  em particular, o sistema de produção Toyota para desenvolvimento  de software. Consiste em sete princípios: eliminar o desperdício,  aumentar a aprendizagem, decidir o mais tarde possível,  entregar o mais rápido possível, capacitar a equipe,  construir integridade, e ver o todo
                    
\item [5. Scrum]
%
centra-se na gestão de projetos em situações em que é difícil  planejar o futuro, com mecanismos de '' controle de processos  empíricos ", onde loops de feedback constituem o elemento central.  Software é desenvolvido por uma equipa de auto-organização  em incrementos (chamado sprints), começando com o planejamento  e terminando com uma retrospectiva. Recursos a serem implementadas  no sistema estão registrados em um backlog . Em seguida,  o proprietário do produto decide quais itens do backlog deve ser  desenvolvido da seguinte sprint. Coordenar membros da equipe  de seu trabalho em uma reunião diária de stand-up. Um membro  da equipe, o scrum master, é responsável pela resolução de  problemas que impedem a equipe de trabalho de forma eficaz.
 
\item [6. Extreme programming (XP)]
concentra-se nas melhores práticas para o desenvolvimento.  Consiste em doze práticas: o jogo de planejamento, pequenos lançamentos,  metáfora, design simples, testes, refatoração, programação em pares,  propriedade coletiva, integração contínua, 40 h semana, os clientes no local,  e os padrões de codificação. A revista XP2 consiste nas seguintes  ''práticas primárias": sentar-se juntos, toda a equipe, trabalho informativo,  o trabalho energizado, programação em pares, histórias, ciclo semanal,  ciclo trimestral, slack, construção de 10 minutos, integração contínua,  testes primeiro que programação e design incremental.  Há também 11 ``corollary practices''.
%\end{enumerate}
\end{description}

\subsection{Princípios Ágeis}
\label{principios-ageis}
O manifesto ágil define 12 princípios que devem ser seguidos~\cite{manifesto2013}
%
\begin{enumerate}
\item Nossa maior prioridade é satisfazer o cliente, através da entrega adiantada e contínua de software de valor.
\item Aceitar mudanças de requisitos, mesmo no fim do desenvolvimento. Processos ágeis se adéquam a mudanças, para que o cliente possa tirar vantagens competitivas.
\item Entregar software funcionando com frequência, na escala de semanas até meses, com preferência aos períodos mais curtos.
\item Pessoas relacionadas a negócios e desenvolvedores devem trabalhar em conjunto e diariamente, durante todo o curso do projeto.
\item Construir projetos ao redor de indivíduos motivados. Dando a eles o ambiente e suporte necessário, e confiar que farão seu trabalho.
\item O Método mais eficiente e eficaz de transmitir informações para, e por dentro de um time de desenvolvimento, é através de uma conversa cara a cara.
\item Software funcional é a medida primária de progresso.
\item Processos ágeis promovem um ambiente sustentável. Os patrocinadores, desenvolvedores e usuários, devem ser capazes de manter indefinidamente, passos constantes.
\item Contínua atenção à excelência técnica e bom design, aumenta a agilidade.
\item Simplicidade: a arte de maximizar a quantidade de trabalho que não precisou ser feito.
\item As melhores arquiteturas, requisitos e designs emergem de times auto-organizáveis.
\item Em intervalos regulares, o time reflete em como ficar mais efetivo, então, se ajustam e otimizam seu comportamento de acordo.
\end{enumerate}

Os princípios ágeis devem ser vividos pela equipe e é o principal motivador e base para o desenvolvimento ágil. As práticas ágeis adotadas serão de acordo com o método escolhido e ajudarão a obter os objetivos propostos pelo método, mas são nos princípios ágeis isso sempre deve ser fundamentado de acordo com o manifesto ágil.



