\chapter{Métodos Empíricos}
\label{cap-met-emp}

%------------------------------------------------------------------------------
%inserir introdução do capítulo
%-----------------------------------------------------------------------------

\section{Software Livre}
\label{sec-swl}

O software, em um sistema computacional, é o componente que contém o conhecimento relacionado aos problemas a que a computação se aplica, contendo diversos aspectos que ultrapassam questões técnicas \cite{meirelles2013metrics}, como por exemplo: 

\begin{itemize}
\item O processo de desenvolvimento de software;
\item Os mecanismos econômicos (gerenciais, competitivos, sociais, cognitivos, etc.) que regem esse desenvolvimento e seu uso;
\item O relacionamento entre desenvolvedores, fornecedores e usuários de software;
\item Os aspectos éticos e legais relacionados ao software;
\end{itemize}

O que define e diferencia o software livre de software proprietário vai do entendimento desses quatro pontos dentro do que é conhecido como \textit{ecossistema do software livre}~\cite{meirelles2013}.
O princípio básico desse ecossistema refere-se ao direito dos usuários de executar, copiar, distribuir, estudar, alterar e melhorar o software. Estas estão definidas nas quatro liberdades para os usuários do software descritas no portal do GNU:

\begin{itemize}
\item A liberdade de executar o programa, para qualquer propósito (liberdade no. 0)~\footnote{Disponível em: \url{http://www.gnu.org/philosophy/free-sw.pt-br.html}, acessado em: Maio de 2014 \label{ft:gnu-ref}};
\item A liberdade de estudar como o programa funciona, e adaptá-lo para as suas necessidades (liberdade no. 1)\footref{ft:gnu-ref}\footnote{Aceso ao código-fonte é um pré-requisito para esta liberdade \label{ft:acessocodfonte}};
\item A liberdade de redistribuir cópias de modo que você possa ajudar ao seu próximo (liberdade no. 2)\footref{ft:gnu-ref};
\item A liberdade de aperfeiçoar o programa, e liberar os seus aperfeiçoamentos, de modo que toda a comunidade se beneficie (liberdade no. 3) \footref{ft:gnu-ref}\footref{ft:acessocodfonte}.
\end{itemize}

Um software é considerado livre quando os usuários deste possuem todas essas liberdades. Assim, você deve ser livre para redistribuir cópias, sejam com ou sem modificações, sem custo associado ou cobrando uma taxa pela distribuição, para qualquer um em qualquer lugar. Além disso, a liberdade de redistribuir cópias deve incluir formas binárias ou executáveis do programa, assim como o código-fonte.

Desenvolvedores ou colaboradores devem ser livres para fazer modificações e usá-las privativamente no seu trabalho ou lazer, sem mencionar a fonte do código. A liberdade de executar o programa significa a liberdade para qualquer tipo de pessoa, física ou jurídica, utilizar o software em qualquer tipo de sistema computacional, para qualquer tipo de trabalho ou finalidade. Uma condição necessária para softwares livres é a liberdade de acesso ao código-fonte {gnu2013}, principalmente no que diz respeito às liberdade número 1 e 3.

%Essa liberdade dentro do software livre é dita, por Richard Stallman, como a principal vantagem. Software não-livre é ruim porque nega sua liberdade. Então, perguntar sobre as vantagens práticas do software livre é como perguntar sobre as vantagens práticas de não estar algemado ~\cite{stallman2009}.

%
O GNU\footnote{GNU é um acrônimo para Gnu is Not Unix, além de ser o nome de um mamífero escolhido por Richard Stallman para batizar um sistema operacional completamente livre} quis dar liberdade aos utilizadores. Contudo precisava usar termos de distribuição que impediriam software livre de ser transformado em software proprietário. O método usado foi chamado de copyleft. Copyleft é um método geral para fazer um software livre e exige que todas as versões modificadas e estendidas do programa sejam também livres. Ele utiliza a lei de direitos autorais, mas veio para servir como oposto de sua finalidade usual: ao invés de um meio de privatizar o software, torna-se um meio de manter o software livre~\cite{stallman2009}.

\section{Processo de Desenvolvimento de Software Livre}
\label{sec-proc-sl}

O aspecto mais importante de um software livre, sob a perspectiva da Engenharia de Software é o seu processo de desenvolvimento. Um projeto de software livre começa quando um desenvolvedor individual ou uma organização decidem tornar um projeto de software acessível ao público. Seu código-fonte é licenciado de forma a permitir seu acesso e alterações subsequentes por qualquer pessoa. Tipicamente, qualquer pessoa pode contribuir com o desenvolvimento, mas mantenedores ou líderes decidem quais contribuições serão incorporadas à release oficial. Não é uma regra, mas projetos de software livre, muitas vezes, recebem colaboração de pessoas geograficamente distantes que se organizam ao redor de um ou mais líderes \cite{corbucci2011freemethods}. 

Há características presentes no software livre que, a princípio, tornam incompatível a aplicação de métodos ágeis em seu desenvolvimento. Entre essas características estão a distância entre os desenvolvedores e a diversidade entre suas culturas, que dificultam a comunicação, um dos principais valores dos métodos ágeis. Entretanto, o sucesso resultante de alguns projetos de software livre, como é o caso do Kernel do Linux \footnote{\url{https://www.kernel.org/}}, fizeram surgir estudos com foco na união dessas duas vertentes.

Analisando um pouco melhor projetos de software livre, é possível notar que esses compartilham princípios e valores presentes no manifesto ágil \footnote{\url{http://agilemanifesto.org/}}. Adaptação a mudanças, trabalhar com \textit{feedback} contínuo, entregar funcionalidades reais, respeitar colaboradores e usuários e enfrentar desafios, são qualidades esperadas em desenvolvedores que utilizam métodos ágeis e são naturalmente encontradas em projetos de software livre.

Num trabalho realizado, \citeonline{corbucci2011freemethods} analisa semelhanças entre projetos de software livre e métodos ágeis, através de uma relação entre os quatro valores enunciados no manifesto ágil e práticas realizadas em projetos livres. 

%
%TODO: descrever relação entre valores do manifesto ágil e práticas de projetos de software livre.
% Não precisa colocar o manifesto ágil vai direto para a relação.
% TODO: Não explicou a questão dos níveis de colaboração, desde o usuário passivo até um desenvolvedor core, exemplificando que, no momento, através deste trabalho, você é um desenvolvedor periférico.
%TODO: Reescrever com suas palavras no TCC 2 (TESE DO PAULO ABAIXO)
%
Conceitualmente, os valores semelhantes são:

\begin{itemize}

\item {Indivíduos e interações são mais importantes que processos e ferramentas.}

\item {Software em funcionamento é mais importante que documentação abrangente.}

\item {Colaboração com o cliente (usuários) é mais importante que negociação de contratos.}

\item {Responder às mudanças é mais importante que seguir um plano.}

\end{itemize}

Além disso, várias práticas disseminadas pelas metodologias ágeis são usadas no
dia-a-dia dos desenvolvedores e equipes das comunidades
de software livre~\cite{corbucci2011freemethods}:

\begin{itemize}

\item {Código compartilhado (coletivo);}
\item {Projeto simples;}
\item {Repositório único de código;}
\item {Integração contínua;}
\item {Código e teste;}
\item {Desenvolvimento dirigido por testes, e}
\item {Refatoração.}

\end{itemize}

Observar e entender esses aspectos nos projetos de software livre tornam-se
relevantes à medida que muitos projetos de software livre não vão além dos
estágios iniciais e muitos acabam sendo abandonados antes de produzir
resultados razoáveis.
%
Isso sugere que, mesmo com o sucesso de alguns projetos de software livre,
as comunidades, com ou sem a participação de empresas, podem avançar no
acompanhamento do desenvolvimento dos projetos de software livre que participam.
%
Olhar o processo de desenvolvimento de software livre do ponto de vista da 
Engenharia de Software e as possíveis sinergias com os métodos ágeis podem
contribuir para um melhor rendimento dessa disposição na criação e colaboração
em torno de projetos de software livre~\cite{meirelles2013metrics}.

Na prática, dentro do processo de desenvolvimento de software livre, após lançar
uma versão inicial e divulgar o projeto, os usuários interessados começam a
usar o software livre em questão. 
%
De acordo com Eric Raymond,``bons programas nascerem de
necessidades pessoais'', esses usuários podem também ser desenvolvedores, que
irão colaborar com o projeto a fim de atenderem às suas próprias necessidades.
%
Destacando a colaboração no código-fonte, essas melhorias são enviadas aos
mantenedores do projeto como \emph{patches}, ou seja, arquivos que
contém as modificações no código e que serão analisados pelos mantenedores que,
caso concordem com a mudança e com a sua implementação em si, irão
aplicá-las ao repositório oficial do projeto.
%
Portanto, mesmo que em projetos maiores outros aspectos sejam levados em consideração ou
sigam processos mais burocrático de colaboração, a essência da colaboração
técnica está no envio e análise de trechos de código-fonte \cite{meirelles2013metrics}.

\section{Padrões de Software Livre}
\label{sec-padroes-sl} 

Um software livre é concebido através de um processo de contribuições, o qual possui características especiais que promovem o surgimento de diversas práticas influenciadas por diversas forças. Tais práticas são conhecidas como padrões de software livre. Para simplificar, nesta seção o termo padrão está associado a padrões de software livre. Esses padrões estão organizados dentro de três grupos:

\begin{itemize}

	\item \textbf{Padrões de seleção} auxiliam prováveis colaboradores a selecionar projetos adequados.

		\begin{itemize}

			\item O primeiro padrão de seleção recomenda colaboradores novatos a "caminhar sobre terreno conhecido", ou seja, se deseja contribuir, começar por algum software que seja familiar, como por exemplo, um browser, editor de texto, IDE\footnote{Integrated Development Environment, um ambiente integrado para desenvolvimento de software}, ou qualquer outro software que já se utiliza.

			\item O segundo padrão é similar ao primeiro, porém ao invés da ferramenta ser familiar, esse padrão recomenda que o colaborador tenha conhecimentos na linguagem ou tecnologia utilizada no projeto.

			\item Já o terceiro padrão desse grupo motiva colaboradores a procurar por projetos de software livre que ofereçam funcionalidades atrativas, mesmo que o novo colaborador não tenha familiaridade com a ferramenta nem com a tecnologia utilizada em seu desenvolvimento.
		\end{itemize}

O terceiro padrão é o que melhor se encaixa ao contexto desse trabalho, já que o Mezuro não era uma ferramenta utilizada no cotidiano e tão pouco familiar. Além disso, as tecnologias utilizadas em seu desenvolvimento não eram as de maior conhecimento. Entretanto, as funcionalidades providas por essa plataforma foi determinante para essa contribuição.

	\item \textbf{Padrões de envolvimento} lidam com os primeiros passos para que o colaborador se familiarize e se envolva com o projeto selecionado.

		\begin{itemize}
			\item Entrar em contato com mantenedores para aprender sobre o contexto 	histórico e político no qual aquele projeto está inserido.

			\item Realizar instalação e checar se todo o ambiente do projeto está 	corretamente configurado em um período limitado de tempo (máximo um dia)

			\item Durante uma apresentação do sistema,  por parte de algum mantenedor, interagir para se familiarizar melhor com funcionalidades e cenários presentes no sistema.

			\item Avaliar o estado do sistema através de uma breve, mas intensa revisão de código. Isso ajuda a ter uma primeira impressão sobre a 					qualidade do código-fonte.

			\item Através da leitura, avaliar a relevância da documentação em um 			período limitado de tempo.

			\item Checar a lista de tarefas a serem feitas. Ela pode conter bons 			pontos de partida para começar uma contribuição.

			\item Relacionado ao padrão mencionado acima, está o padrão que recomenda novos colaboradores iniciarem por tarefas mais fáceis. Começar uma tarefa 			e termina-la é importante para manter colaboradores motivados, e conforme 			ganharem mais experiencia e familiaridade com o software avançam para 			tarefas mais complexas.
	\end{itemize}

No contexto deste trabalho, muitos dos padrões desse grupo foram inseridos ao processo de contribuição. Por exemplo, o orientador deste trabalho é também mantenedor da plataforma Mezuro, assim como outros colaboradores da plataforma, auxiliaram durante o processo de envolvimento, apresentando funcionalidades e principais cenários do sistema, além de fornecer documentação necessária para o entendimento do histórico e contexto no qual o Mezuro está inserido.
	
	\item \textbf{Padrões de contribuição} documenta as melhores práticas para 	se contribuir com softwares livres. Os grupos anteriores tratavam como iniciar 	e se familiarizar com um projeto de software livre. Esse grupo, por sua vez, 		contém padrões que auxiliam o fornecimento de insumos para projetos de software livre, seja código-fonte ou outros artefatos presentes no processo de desenvolvimento.

	\begin{itemize}

		\item Uma boa contribuição para projetos de software em geral, é a escrita de documentação. O código-fonte muitas vezes não é o suficiente para que todos os envolvidos entendam o andamento do projeto, pois apesar de promoverem o software não possuem conhecimento técnico suficiente. Além disso, documentação do projeto auxilia na manutenção e evolução do produto.

		\item Muitos softwares livres não suportam o idioma de diversos colaboradores. Um bom ponto de partida seria a internacionalização do sistema, incluindo a própria linguagem no sistema.

		\item Reportar bugs eficientemente, pois é comum que colaboradores identifiquem bugs mas ao reporta-los não são claros com respeito ao seu contexto, dificultado sua correção.

		\item Utilizar a versão correta para tarefas. Durante o desenvolvimento de software há diferentes versões, onde há no mínimo uma versão estável e uma versão desenvolvimento. É recomendado utilizar a versão estável para reportar bugs e a versão de desenvolvimento para implementar novas funcionalidades e tudo que não está relacionado com correção de defeitos existentes.

		\item Separar alterações não relacionadas. Se tratando de sistemas de controle de versão\footnote{SCM - Source Code Management - GIT, SVN, Baazar, Mercurial, entre outros} há uma ação conhecida como commit, onde as alterações realizadas são agrupadas e gravadas. É recomendado que num mesmo commit as alterações sejam relacionadas.

		\item Mensagens de commit explicativas para facilitar o entendimento e identificação do que foi desenvolvido ou alterado para o restante dos colaboradores.

		\item Documentar as próprias modificações. Desenvolvedores, geralmente, alteram o código, corrigem bugs, adicionam novas funcionalidades, mas não atualizam a documentação, a qual se torna desatualizada. Por isso é recomendado documentar as alterações antes de submete-as ao repositório.

		\item Manter-se atualizado com o estado atual do projeto, ajudando a evitar duplicação de esforços e identificar oportunidades de colaboração. Isso é importante pois um projeto de software livre é um esforço coletivo, mas às vezes é difícil coordenar o esforço de pessoas com diferentes horários e prioridades.

	\end{itemize}

\end{itemize}

%TODO: Faltou um sincronização deste padrões com o seu trabalho.

Em resumo, esses padrões não são regras, apenas indicam um bom caminho para contribuições. Por exemplo, o software tratado neste trabalho, o autor principal do mesmo, não era familiar no início do processo de contribuição para a colaboração da evolução de um software livre.

\section{Métodos Ágeis}
\label{sec-metedos-ageis}

Métodos ágeis (AM) é uma coleção de metodologias baseada na prática para modelagem efetiva de sistemas baseados em software. É uma filosofia onde muitas metodologias se encaixam. As metodologias ágeis aplicam uma coleção de práticas, guiadas por princípios e valores que podem ser aplicados por profissionais de software no dia a dia~\cite{manifesto2001}%Procurar referencia.

\subsection{Metodologia Ágil}
\label{metodologia-agil}

A expressão "Metodologias Ágeis" tornou-se popular em 2001 quando dezessete especialistas em processos de desenvolvimento de software representando diversos métodos como Scrum ~\cite{schwaber2002},
Extreme Programming (XP) e outros, estabeleceram princípios comuns compartilhados por todos esses métodos. Foi então criada a Aliança Ágil e o estabelecimento do Manifesto Ágil ~\cite{manifesto2001}
Os conceitos chave do Manifesto Ágil são:
\textbf{Indivíduos e interações} ao invés de processos e ferra-
mentas.
\textbf{Software executável} ao invés de documentação.
\textbf{Colaboração do cliente} ao invés de negociação de contratos.
\textbf{Respostas rápidas a mudanças} ao invés de seguir planos.
O Manifesto Ágil não discrimina processos, ferramentas, documentação, negociação de contratos ou o planejamento, mas simplesmente mostra que eles têm importância secundária quando comparado com os indivíduos, interações, software executável, participação do cliente e feedback rápido a mudanças e alterações.

%
Tory Dyba (\citeyear{dyba2008}) em uma revisão sistemática elenca seis dos principais métodos ágeis presente em 2008 e trás uma breve descrição de cada um, o que ajuda a observar características das comunidades ágeis já que os principais especialistas desses métodos ajudaram a escrever o manifesto ágil. Eles são:

\begin{description}
%\begin{enumerate}

\item [1. Crystal Clear]
%
centra-se na comunicação de pequenas equipes de
desenvolvimento de software que não possui ciclo de vida crítico.
%
Seu desenvolvimento tem sete características: entrega frequente,
melhoria reflexivo, comunicação osmótica, segurança pessoal, foco,
fácil acesso a usuários experientes e os requisitos para o ambiente técnico.

\item [2. Dynamic software development method (DSDM)]
%
divide projetos em três fases: pré-projeto, o ciclo de vida do projeto  e pós projeto. Nove princípios subjacentes ao DSDM: o envolvimento do usuário, capacitando da equipe do projeto,  a entrega frequente, atender às necessidades de negócios atuais,  desenvolvimento iterativo e incremental, permitir mudanças,  o escopo de alto nível a ser fixados antes do início do projeto,  testar todo o ciclo de vida e eficiente e eficaz comunicação.
            
\item [3. Feature-driven development]
%
combina desenvolvimento model-driven e ágil, com ênfase  em modelo inicial de objeto, a divisão do trabalho em funções  e design iterativo para cada recurso. Afirma ser adequado para  o desenvolvimento de sistemas críticos. Uma iteração de um  recurso é composto de duas fases: concepção e desenvolvimento
       
\item [4. Lean software development]
%
uma adaptação de princípios de produção de carne magra e,  em particular, o sistema de produção Toyota para desenvolvimento  de software. Consiste em sete princípios: eliminar o desperdício,  aumentar a aprendizagem, decidir o mais tarde possível,  entregar o mais rápido possível, capacitar a equipe,  construir integridade, e ver o todo
                    
\item [5. Scrum]
%
centra-se na gestão de projetos em situações em que é difícil  planejar o futuro, com mecanismos de '' controle de processos  empíricos ", onde loops de feedback constituem o elemento central.  Software é desenvolvido por uma equipa de auto-organização  em incrementos (chamado sprints), começando com o planejamento  e terminando com uma retrospectiva. Recursos a serem implementadas  no sistema estão registrados em um backlog . Em seguida,  o proprietário do produto decide quais itens do backlog deve ser  desenvolvido da seguinte sprint. Coordenar membros da equipe  de seu trabalho em uma reunião diária de stand-up. Um membro  da equipe, o scrum master, é responsável pela resolução de  problemas que impedem a equipe de trabalho de forma eficaz.
 
\item [6. Extreme programming (XP)]
concentra-se nas melhores práticas para o desenvolvimento.  Consiste em doze práticas: o jogo de planejamento, pequenos lançamentos,  metáfora, design simples, testes, refatoração, programação em pares,  propriedade coletiva, integração contínua, 40 h semana, os clientes no local,  e os padrões de codificação. A revista XP2 consiste nas seguintes  ''práticas primárias": sentar-se juntos, toda a equipe, trabalho informativo,  o trabalho energizado, programação em pares, histórias, ciclo semanal,  ciclo trimestral, slack, construção de 10 minutos, integração contínua,  testes primeiro que programação e design incremental.  Há também 11 ``corollary practices''.
%\end{enumerate}
%\caption{Métodos ágeis, extraído da revisão sistemática \cite{dyba2008}}
\end{description}
\begin{flushright}
Métodos ágeis, extraído da revisão sistemática ~\cite{dyba2008}
\end{flushright}

Alguns desses métodos ainda são amplamente utilizados, discutidos e contribuem para o entendimento do desenvolvimento ágil, pois definem práticas que ajudam a instigar a vivência ágil dentro da equipe. É o método escolhido que caracteriza as práticas a serem seguidas.

\subsection{Princípios Ágeis}
\label{principios-ageis}
O manifesto ágil define 12 princípios que devem ser seguidos~\cite{manifesto2001}
%
\begin{enumerate}
\item Nossa maior prioridade é satisfazer o cliente, através da entrega adiantada e contínua de software de valor.
\item Aceitar mudanças de requisitos, mesmo no fim do desenvolvimento. Processos ágeis se adéquam a mudanças, para que o cliente possa tirar vantagens competitivas.
\item Entregar software funcionando com frequência, na escala de semanas até meses, com preferência aos períodos mais curtos.
\item Pessoas relacionadas a negócios e desenvolvedores devem trabalhar em conjunto e diariamente, durante todo o curso do projeto.
\item Construir projetos ao redor de indivíduos motivados. Dando a eles o ambiente e suporte necessário, e confiar que farão seu trabalho.
\item O Método mais eficiente e eficaz de transmitir informações para, e por dentro de um time de desenvolvimento, é através de uma conversa cara a cara.
\item Software funcional é a medida primária de progresso.
\item Processos ágeis promovem um ambiente sustentável. Os patrocinadores, desenvolvedores e usuários, devem ser capazes de manter indefinidamente, passos constantes.
\item Contínua atenção à excelência técnica e bom design, aumenta a agilidade.
\item Simplicidade: a arte de maximizar a quantidade de trabalho que não precisou ser feito.
\item As melhores arquiteturas, requisitos e designs emergem de times auto-organizáveis.
\item Em intervalos regulares, o time reflete em como ficar mais efetivo, então, se ajustam e otimizam seu comportamento de acordo.
\end{enumerate}

Os princípios ágeis devem ser vividos pela equipe e é o principal motivador e base para o desenvolvimento ágil. As práticas ágeis adotadas serão de acordo com o método escolhido e ajudarão a obter os objetivos propostos pelo método, sempre levando em consideração os princípios ágeis definidos pelo manifesto ágil.



