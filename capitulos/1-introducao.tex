\chapter{Introdução}
\label{cap-introducao}

Atualmente, as tecnologias da informação exercem cada vez mais influência na sociedade, seja na interação entre pessoas, ou nas relações que empresas possuem com o mercado. Nesse sentido, sistemas de software tem recebido mais atenção, dado que cada vez mais podem determinar o sucesso ou fracasso nessas relações.

Por volta da década de 60, o software era comercializado juntamente com hardware. O código-fonte era disponibilizado junto com o software e muitos usuários compartilhavam código e informações. O software era dito livre até a década de 70, quando a International Business Machines (IBM) decidiu comercializar seus programas separados do hardware. Com isso as restrições de acesso ao código-fonte se tornaram comuns, assim seu compartilhamento se tornava cada vez mais escasso, surgiram assim os softwares proprietários.

Durante a década de 80, porém, o contexto de software livre que permeava o início do software ressurgiu, em suma, por iniciativa Richard Stallman. Em 1983, após uma experiência negativa com softwares proprietários, ele deu origem ao Projeto GNU. Stallman objetivava criar um mecanismo para garantir direitos de cópia, modificação e redistribuição de software. Com isso surgiu a Licença General Public License (GPL).

A definição de software é a mesma, seja ele proprietário ou livre. Softwares são constituídos por um conjunto de procedimentos, dados, possível documentação, e satisfazem necessidades específicas de determinados usuários. Além da definição, os dois tipos de software mencionados aqui possuem também a necessidade de serem mantidos e evoluídos.

Um exemplo de software que passa por um processo de evolução é a plataforma Mezuro, uma ferramenta para monitorar código-fonte. O Mezuro é desenvolvido como um projeto de software livre, que a princípio era um \emph{plugin} de uma plataforma de desenvolvimento de redes sociais denominada Noosfero. Porém, o Mezuro cresceu em tamanho e complexidade, motivando a reescrita do seu código-fonte, transformando-o em uma plataforma independente do Noosfero, aumentando a consistência em relação ao seu propósito, que é análise e interpretação de código-fonte. Além disso,  para tornar o Mezuro mais competitivo em relação às ferramentas semelhantes e ampliar o interesse de utilização por parte de usuários, decidiu-se evoluí-lo.

A necessidade de ampliar o interesse por parte do usuário dentro do projeto Mezuro, advém da crescente preocupação com a usabilidade que o desenvolvimento de software livre vem adquirindo nos últimos anos, ainda que essa preocupação esteja normalmente limitada à projetos de grande visibilidade, geralmente patrocinados por grandes empresas \cite{nichols2006}.
%
Ao lado da própria dificuldade em se mensurar objetivamente a usabilidade de um sistema, é comum em programas de software livre que haja pouco incentivo (ou interesse) nesse aspecto, dado que a prioridade do mesmo é a implementação das funcionalidades. Essa cultura leva o desenvolvedor a iniciar o projeto pelo código, deixando o \emph{design} de interfaces em segundo plano  \cite{thomas2008}.

A fim de evitar esse cenário dentro do projeto Mezuro, um ciclo de usabilidade foi integrado dentro do processo de evolução da plataforma. Para poder inserir uma maior preocupação com a usabilidade dentro de um projeto de desenvolvimento livre com práticas ágeis foi necessário um levantamento de técnicas de usabilidade ágeis para ter o menor impacto dentro do ciclo de vida do projeto, gerando assim, colaboradores com foco na usabilidade para evolução das camadas de \textit{front-end} e \textit{back-end}, bem como, dessa forma, disseminar o interesse da melhoria desse requisito não funcional -- a usabilidade.

Um dos desafios inerentes ao desenvolvimento de software é a manutenção e aumento da qualidade do sistema desenvolvido. Muitos indícios da qualidade de um software são encontrados no seu código-fonte através de suas métricas. Porém, quanto maior a quantidade de métricas extraídas mais difícil é a leitura e interpretação das mesmas.
%
Visando maximizar o poder de compreensão do usuário em relação as métricas ou dados apresentados, um dos aspectos referentes a interação do usuário com a ferramenta é a visualização de informações, que tem como objetivo exatamente auxiliar o usuário a interpretar os valores apresentados, por meio de representações gráficas e interativas apoiadas por computador \cite{rafaelmessiasmartins2012}.

\section{Objetivos}

\subsection{Objetivos Gerais}

Neste trabalho de conclusão de curso, há como principal objetivo colaborar com evolução da plataforma de monitoramento de código-fonte Mezuro, de forma que migre do cenário de ser um plugin para torna-se uma aplicação independente.

\subsection{Objetivos Específicos}

Como objetivos específicos, este trabalho visa:

\begin{enumerate}

Reescrever o código do background do Mezuro;

\item Reescrever o código do \textit{background}\footnote{Sequência de passos ou processo que executa em paralelo e ``por trás'' da camada de visualização sem a intervenção do usuário} do Mezuro (modelo e controladores);
\item Reescrever o código do \textit{front-end} do Mezuro (apresentação);
\item Verificar resultados da aplicação de uma técnica de usabilidade ágil em uma comunidade de software livre;
\item Comparar a nova arquitetura com a antiga;
\item Colaborar para a evolução do Mezuro, segundo os padrões de contribuição para projetos de Software Livre;

\section{Metodologia}

A pesquisa é utilizada para a melhoria da qualidade do contexto pesquisado através do novo conhecimento adquirido, assim nesse trabalho quanto a natureza da pesquisa foi abordado a pesquisa aplicada ou tecnológica. De acordo com Barros e Lehfeld \citeyear{barros2000}, a pesquisa aplicada possui como motivação a necessidade de gerar conhecimento para aplicação de seus resultados, com o objetivo de “contribuir para fins práticos, visando à solução mais ou menos imediata do problema encontrado na realidade”. Já que um estopim já havia sido dado pelo trabalho de pesquisa fundamental realizado na dissertação de mestrado "Aplicação de práticas de usabilidade ágil em software livre” \cite{santos2012} e com esse conhecimento gerado resolveu-se aplicá-lo a fim de se obter uma melhoria na evolução do projeto Mezuro.

Quanto aos objetivos, a pesquisa exploratória vai em conjunto com a necessidade de se verificar um padrão e/ou ideias em relação a introdução de um ciclo de usabilidade dentro de um projeto de desenvolvimento de software livre com práticas ágeis, já que esse é um assunto com poucos estudos a respeito. Associado a pesquisa exploratória foi realizado uma revisão sistemática a fim de se verificar os padrões existentes que poderiam ser aplicados para inserir a perspectiva de usabilidade em um contexto de projetos ágeis e de software livre.

Quanto aos procedimentos foi realizado um estudo de caso, esse um dos focos principais motivado pela evolução da plataforma Mezuro, onde se poderia aplicar as ideias propostas. A pesquisa foi realizado em associação aos laboratórios da Universidade de São Paulo (USP) e UnB (Universidade de Brasília) Gama. Os resultados desse estudo forneceram dados qualitativos para a análise.

O primeiro passo para realização deste trabalho foi o contato com a equipe que mantém a plataforma Mezuro, que se deu através do professor Dr. Paulo Meirelles, um dos mantenedores. Como muitos projetos de software livre contam com desenvolvedores distantes geograficamente, com o Mezuro não foi diferente. Após o contato com o primeiro mantenedor, iniciou-se os pareamentos remotos, através de vídeo-conferências, para maior familiaridade com o código e funcionalidades fornecidas pela ferramenta.

Após a apresentação geral de todo o código e funcionalidades do Mezuro, iniciou-se o treinamento na ferramenta utilizada no desenvolvimento do Mezuro, o arcabouço \textit{Ruby on Rails}. Essa fase aconteceu simultaneamente ao desenvolvimentos das funcionalidades reservadas a equipe da Universidade de Brasília.

Para o desenvolvimento dessas funcionalidades, foram utilizadas práticas ágeis como \textit{pair programming}\footnote{Dois programadores, piloto e co-piloto, trabalham juntos na mesma estação de trabalho. O piloto codifica, enquanto o co-piloto acompanha e auxilia o piloto na tomada das melhores decisões}, divisão das iterações em \textit{sprints}, definição das unidades de trabalho em \textit{backlog}, reuniões de retrospectiva para análise dos resultados produzidos ao fim de uma \textit{sprint}, além das reuniões constantes que satisfazem um dos princípios básicos da metodologia ágil que é a comunicação.

\section{Organização do Trabalho}

Em paralelo à interação e colaborações ao código do Mezuro, houve a escrita deste trabalho, que está organizado em capítulos. O Capítulo 2 apresenta os principais conceitos de software livre, métodos ágeis, padrões de software livre e processo de desenvolvimento de software. O Capítulo 3 trás os principais, dos diversos conceitos, de evolução de projeto de software divido em duas partes, evolução de software e evolução da arquitetura. No Capítulo 4 é introduzido a abordagem da experiência do usuário focada na usabilidade com aspectos de melhoria da interface e na visualização de software sendo essas requisitos não funcionais dentro do projeto. Finalmente, o Capítulo 5 apresenta com mais detalhes a plataforma Mezuro, seu processo de desenvolvimento, fatores que motivaram a reescrita de seu código, assim como o processo e resultados gerados da aplicação de um ciclo de evolução de software focado na usabilidade tanto em \textit{front-end} quanto em \textit{back-end} na plataforma Mezuro. O Capítulo 6 é a conclusão do trabalho, trazendo limitações, trabalhos futuros e como a plataforma mezuro poderá continuar com o legado deixado pelo trabalho realizado.

