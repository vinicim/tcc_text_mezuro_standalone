\chapter{Introdução}

Atualmente as tecnologias da informação exercem cada vez mais influência na sociedade, seja na interação entre pessoas, ou nas relações que empresas possuem com o mercado. Nesse sentido, sistemas de software tem recebido mais atenção, dado que cada vez mais podem determinar o sucesso ou fracasso nessas relações.

%Monopólio dos software proprietários e início dos softwares livre
Por volta da década de 60 o software era comercializado juntamente com hardware. O código-fonte era disponibilizado junto com o software e muitos usuários compartilhavam código e informações. O software era dito livre até a década de 70, quando a IBM\footnote{Internacional Business Machines é uma empresa americana da área de tecnologia da informação. Comercializa hardware e software além de serviços de hospedagem e infra-estrutura.} decidiu comercializar seus programas separados do hardware. Com isso as restrições de acesso ao código-fonte se tornaram comuns, assim seu compartilhamento se tornava cada vez mais escasso, surgiram assim os softwares proprietários.

Durante a década de 80, porém, o contexto de software livre que permeava o início do software  ressurgiu graças a Richard Stallman. Em 1983, após uma experiência negativa com softwares proprietários, ele deu origem ao Projeto GNU\footnote{Projeto que tem como objetivo a criação de um sistema operacional livre}. Stalman objetivava criar um mecanismo para garantir direitos de cópia, modificação e redistribuição de software. Com isso surgiu a Licença GPL\footnote{General Public Lincense ou Licensa Pública Geral}.

A definição de software é a mesma, seja ele proprietário ou livre. Softwares são constituídos por um conjunto de procedimentos, dados, possível documentação, e satisfazem necessidades específicas de determinados usuários. Além da definição, os dois tipo de software mencionados aqui possuem também a necessidade de serem mantidos e evoluídos.

A evolução de software começou a receber mais atenção a partir dos estudos realizados por Meir M. Lehman no final da década de 60, assim se tornou uma área de conhecimento da engenharia de software. A partir de seus estudos Lehman elaborou um conjunto de oito leis que são conhecidas "Leis de Lehman", as quais definem um padrão para evolução de softwares. 

A evolução de software é importante para manter o software consistente com seu ambiente, competitivo em relação a sotwares concorrentes e manter os usuários interessados. Além disso, quando tratamos de software livre a evolução é essencial para manter a comunidade estabelecida ao seu redor, contribuindo para manter esse projeto ativo.  

Um dos desafios inerentes ao desenvolvimento de software é a manutenção e aumento da qualidade do sistema desenvolvido. Muitos indícios da qualidade de um software são encontrados no seu código-fonte através de suas métricas. Porém, extrair métricas de código-fonte manualmente não é uma tarefa considerada pelas equipes de desenvolvimento dado o esforço necessário para tal. Para isso existem ferramentas que auxiliam as equipes de desenvolvimento. Entre essas ferramentas está a plataforma Mezuro, utilizada para o monitoramento de códigos-fonte.

O Mezuro é desenvolvido como um projeto de software livre, que a princípio era um plugin de uma plataforma de desenvolvimento de redes sociais denominada Noosfero. Porém, o Mezuro cresceu muito em tamanho e complexidade desde que foi concebido. Por isso a equipe decidiu pela reescrita de seu código, o transformando em uma plataforma independente foram do Noosfero, visando mante-la consitente com seu ambiente e com seus propósitos, competivivo em relação a ferramentas semelhantes e ampliar o interesse de utilização por parte de usuários. 

\section{Objetivos}

\subsection{Objetivos Gerais}

\subsection{Objetivos Específicos}

\section{Organização do Trabalho}
