\chapter{Software Livre}

Como este trabalho visa contribuir diretamente com um software livre, neste capítulo serão apresentados os principais conceitos relacionados com esse tipo de software. Em um primeiro momento serão apresentadas definições básicas sobre software livre. Na seção...%descrição das seções deste capitulo: padroes contribuição, metodologias ageis e software livre

Um software é constituído por um conjunto de procedimentos, dados, possível documentação, e satisfaz necessidades específicas de determinados usuários. O software é um componente de um sistema computacional de interesse geral, uma vez que vários aspectos relacionados a ele ultrapassam questões técnicas \cite{meirelles2013metrics}, como por exemplo: 

\begin{itemize}
\item O processo de desenvolvimento de software;
\item Os mecanismos econômicos (gerenciais, competitivos, sociais, cognitivos, etc.) que regem esse desenvolvimento e seu uso;
\item O relacionamento entre desenvolvedores, fornecedores e usuários de software;
\item Os aspectos éticos e legais relacionados ao software;
\end{itemize}

%Diferenciar software livre de software proprietario
O entendimento desses quatro pontos é o que diferencia softwares ditos proprietários dos softwares livres, além de definir o que é conhecido como ecossitema do software livre. O princípio básico desse ecossistema é promover a liberdade do usuário, sem discriminar quem tem permissão para usar um software e seus limites de uso, baseado na colaboração e num processo de desenvolvimento aberto \cite{meirelles2013metrics}.

Software livre representa uma classe de sistemas de software, os quais são distribuídos sob licenças cujos termos permitem aos seus usuários utilizar, estudar e modificar e redistribuir o software \cite{terceiro2012freesoftware}. 

\section{Processo de Desenvolvimento de Software Livre}

O aspecto mais importante de um software livre, sob a perspectiva da Engenharia de Software é o seu processo de desenvolvimento. Um projeto de software livre começa quando um desenvolvedor individual ou uma organização decidem tornar um produto de software acessível ao público. Seu código-fonte é licenciado de forma a permitir seu acesso e alterações subsequentes por qualquer pessoa. Tipicamente, este tipo de projeto recebe a colaboração de pessoas geograficamente distantes que se organizam ao redor de um ou mais líderes \cite{corbucci2011freemethods}. 

Mesmo com características presentes no software livre, que a princípio tornam incompatível a aplicação de métodos ágeis nesses projetos. Entre essas características estão a distância entre os desenvolvedores e a diversidade entre suas culturas que dificultam a comunicação, um dos principais valores dos métodos ágeis. Porém projetos de software livre compartilham princípios e valores presentes no manifesto ágil \footnote{Manifesto Ágil}. Esses princípios compartilhados e a crescente adoção de metodologias ágeis resulta no sucesso de vários projetos de software livre, como é o caso do desenvolvimento do kernel do Linux \footnote{\url{https://www.kernel.org/}}. %Por esse motivo, é normal procurar reunir fatores de sucesso de cada caso, assim como 

Um software livre tem a característica do compartilhamento do seu código-fonte. Essa característica oferece vantagens em relação ao software proprietário, que não disponibiliza seu código-fonte. O compartilhamento permite a simplificação de aplicações personalizadas, já que não necessitam serem codificadas do zero, podendo se basear em soluções existentes. Outra vantagem é a melhoria da qualidade \cite{Raymond, 1999}, por conta da grande quantidade de colaboradores, que com diferentes perspectivas e necessidades, propõem melhorias para o sistema, além de identificar e corrgir bugs com mais rapidez.



