\chapter{Software Livre}

%1. SOFTWARE LIVRE COMO - MÉTODOS ÁGEIS
%2. MAPEAMENTO ENTRE PRÁTICAS ÁGEIS E DA COMUNIDADE SW LIVRE (PROCURAR DISSERTAÇÃO DE MESTREDO HUGO %{DISSERTAÇÃO PROF. PAULO})


%Falar das vantagens do software livre
%Licenças
%Padroes (patterns) para software livre
%O que é software
Um software é constituído por um conjunto de procedimentos, dados, e possível documentação. Sua aplicação supre necessidades específicas de dados usuários. O software é um componente de um sistema computacional de interesse geral, uma vez que vários aspectos relacionados a ele ultrapassam questões técnicas \cite{meirelles2013metrics}, como por exemplo: 

\begin{itemize}
\item O processo de desenvolvimento de software;
\item Os mecanismos econômicos (gerenciais, competitivos, sociais, cognitivos, etc.) que regem esse desenvolvimento e seu uso;
\item O relacionamento entre desenvolvedores, fornecedores e usuários de software;
\item Os aspectos éticos e legais relacionados ao software;
\end{itemize}

%Diferenciar software livre de software proprietario
O entendimento desses quatro pontos é o que diferencia softwares ditos restritos dos softwares livres, além de definir o que é conhecido como ecossitema do software livre. O princípio básico desse ecossistema é promover a liberdade do usuário, sem discriminar quem tem permissão para usar um software e seus limites de uso, baseado na colaboração e num processo de desenvolvimento aberto. Software livre é aquele que permite ao usuário estudá-lo, modifica-lo e redistribuí-lo em geral, sem restrições para tal e prevenindo que nao sejam aplicadas restriçoes aos futuros usuários \cite{meirelles2013metrics}.

Um software livre tem a característica de possibilitar o compartilhamento do seu código-fonte. Essa característica oferece vantagens ao software livre em relação ao software restrito, já que esse nao disponibiliza seu código-fonte. O compartilhamento permite a simplificação de aplicações personalizadas, já que não necessitam serem codificadas do zero, podendo se basear em soluções existentes. Outra vantagem é a melhoria da qualidade \cite{Raymond, 1999}, por conta da grande quantidade de colaboradores, que com diferentes perspectivas e necessidades, propem melhorias para o sistema, além de identificar e corrgir bugs com mais rapidez.

