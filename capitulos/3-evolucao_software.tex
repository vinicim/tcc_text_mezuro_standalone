\chapter{Evolução de Software}

%Possíveis tópicos (Baseados no livro: Software Maintenance – Franz Wotowa):

%Motivação
%Definição
%Classificação de Mudanças
%O Processo de Manutenção
%Leis de Lehman

Atualmente as tecnologias da informação exercem cada vez mais influência na sociedade, seja na interação entre pessoas, ou nas relações que empresas possuem com o mercado. Empresas, que possuem parte dos seus lucros associados diretamente, ou não, há sistemas de software, precisam evolui-los, seja para adequa-los à mudanças no ambiente onde estão inseridos, ou para mante-los competitivos frente aos concorrentes. Além desses fatores, quando os sistemas em questão são desenvolvidos como softwares livres, eles também precisam evoluir para que se mantenham sempre atrativos, motivando a comunidade estabelecida ao seu redor. Sistemas estagnados desmotivam usuários ou colaboradores, o que significa risco de perda de mercado ou enfraquecimento de um projeto de software livre, já que esses são feitos de colaboradores
%[modificar esse parágrafo com base no artigo challanges_sw_evolution]

Por outro lado, a manutenção desses sistemas é difícil, consome bastante tempo e recursos. Tarefas como adicionar novas funcionalidades, suporte a novos dispositivos de hardware, correção de defeitos, entre outros, se tornam mais difíceis conforme o sistema cresce e envelhece \cite{godfrey2000evolution}.

Até agora foram mencionados os termos manutenção e evolução de software. Na maioria das vezes esses palavras aparecem juntas na literatura, e embora se refiram ao mesmo fenômeno, possuem ênfases diferentes. Manutenção é o ato de manter uma entidade num estado de reparo, capacidade ou disponibilidade, prevenindo-a contra falhas, mantendo a satisfação dos envolvidos ao longo do ciclo de vida do software. Já a evolução refere-se a um processo de mudança contínuo de um estado mais baixo, simples ou pior para um estado mais alto, mais complexo e melhor, refletindo a soma de todas as alterações implementadas no sistema.

%evolução de software sempre existiu, porem nao era estudada
A evolução de software foi identificada pela primeira vez no final dos anos 60, embora não denominada evolução até 1969, quando Meir M. Lehman realizou um estudo com a IBM, com a ideia de melhorar sua efetividade de programação. Apesar de não ter recebido tanta atenção e pouco impactado nas práticas de desenvolvimento dessa companhia , esse estudo fez surgir um novo campo de pesquisa, a evolução de software.
%\cite{Artigo IBM}.

Com esse estudo Lehman publicou as leis da evolução software:
%melhorar a introdução para as leis de lehman
\begin{itemize}
\item Mudança contínua. Um software deve ser continuamente adaptado, caso contrário se torna progressivamente menos satisfatório.
\item Complexidade crescente. À medida que um software é alterado, sua complexidade cresce, a menos que um trabalho seja feito para mantê-la ou diminuí-la.
\item Auto-regulação. O processo de evolução de software é auto-regulado próximo à distribuição normal com relação às medidas dos atributos de produtos e processos.
\item Conservação da estabilidade organizacional. A não ser que mecanismos de retro-alimentação tenham sido ajustados de maneira apropriada, a taxa media de atividade global efetiva num software em evolução tende a ser manter constante durante o tempo de vida do produto.
\item Conservação da Familiaridade. De maneira geral, a taxa de crescimento incremental e taxa crescimento a longo prazo tende a declinar.
\item Crescimento contínuo. O conteúdo funcional de um software deve ser continuamente aumentado durante seu tempo de vida para para manter a satisfação do usuário.
\item Qualidade decrescente. A qualidade do software será entendida como declinante a menos que o software seja rigorosamente adaptado às mudanças no ambiente operacional.
\item Sistema de Retro-alimentação. Processos de evolução de software são sistemas de retro-alimentação em múltiplos níves, em múltiplos laços (loops) e envolvendo múltiplos agentes.
\end{itemize}

%importancia
A partir desse estudo, a evolução de software passou a receber cada vez mais atenção, e se tornou bastente relevante,  pois quando inserida ou considerada nos processos de desenvolvimento ela resulta numa excelente alternativa para evitar os sintomas do envelhecimento e inconsistencias entre o próprio software e o ambinte onde está inserido \cite{mens2005challenges}. 

Ao contrário das engenharias tradicionais, a engenharia de software tem em mãos um produto abstrato e intangível, o que resulta em alguns desafios inerentes aos processo de desenvolvimento.

\begin{itemize}
\item Manter e melhorar a qualidade do software;
\item Suportar evolução do modelo de desenvolvimento (não só código-fonte);
\item Manter consistência entre artefatos relacionados;
\item Integrar mudanças dentro do ciclo de desenvolvimento de software;
\item Necessidades de bons sistemas de controle de versão;
\item Integração e análise de dados de várias fontes (relatórios de erros, métricas, solicitações de mudança);
\end{itemize}

Mais uma justificativa para a relevancia dos estudos no campo da evolução de software é que ela busca amenizar ou superar os desafios acima \cite{mens2005challenges}.

\section{Evolução de Software Livre}





