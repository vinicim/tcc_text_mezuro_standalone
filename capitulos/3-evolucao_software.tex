\chapter{Evolução de Software}

%Tenho dúvidas na separação desses dois conceitos (manutenção e evolução) do trabalho desenvolvido com o Mezuro. Vejo o trabalho com o Mezuro como um processo de evolução. Por outro lado, sempre vejo evolução atrelada a manutenção na literatura. Assim como o propósito da evolução do Mezuro se encaixa no propósito de uma evolução preventiva, facilitar a manutenbilidade.

%Possíveis tópicos (Baseados no livro: Software Maintenance – Franz Wotowa):

%Motivação
%Definição
%Classificação de Mudanças
%O Processo de Manutenção
%Leis de Lehman

Atualmente as tecnologias da informação exercem cada vez mais influência na sociedade, seja na interação entre pessoas, ou nas relações que empresas possuem com o mercado. Empresas, que possuem parte dos seus lucros associados diretamente, ou não, há sistemas de software, precisam evolui-los, seja para adequa-los à mudanças no ambiente onde estão inseridos, ou para mante-los competitivos frente aos concorrentes. Além desses fatores, quando os sistemas em questão são desenvolvidos como softwares livres, eles também precisam evoluir para que se mantenham sempre atrativos, motivando a comunidade estabelecida ao seu redor. Sistemas estagnados desmotivam usuários ou colaboradores, o que significa risco de perda de mercado ou enfraquecimento de um projeto de software livre, já que esses são feitos de colaboradores.

Já a manutenção desses sistemas é difícil, consome bastante tempo e recursos. Tarefas como adicionar novas funcionalidades, suporte a novos dispositivos de hardware, correção de defeitos, entre outros, se tornam mais difíceis conforme o sistema cresce e envelhece \cite{godfrey2000evolution}.

Até agora foram mencionados os termos manutenção e evolução de software. Na maioria das vezes esses palavras aparecem juntas na literatura, remetendo a ideia que compartilham o mesmo conceito. Porém elas possuem significados diferentes. Manutenção é o ato de manter uma entidade num estado de reparo, capacidade ou disponibilidade, prevenindo-a contra falhas. Já a evolução de um software se refere a um processo de mudança contínuo de um estado mais baixo, simples ou pior para um estado mais alto, mais complexo e melhor.

